% ============================================================================ %
% Encoding: UTF-8 (žluťoučký kůň úpěl ďábelšké ódy)
% ============================================================================ %

% ============================================================================ %
\nn{Úvod}
Tato diplomová práce je zaměřena na~požadavky dispečinkového systému a~to na~modul zpoždění.

V~teoretické části je popsáno jaké prostředky jsou v~této práci používány pro~práci a~vývoj.

Ve~třetí kapitole analytické části je popisován systém diferenční analýzou. V~dalších podkapitolách je popsán současný stav a~nový návrh vylepšení stávajícího modulu zpoždění. Jsou zde návrhy variant cílového stavu, jejich vyhodnocení a~výběr nejvhodnějšího z~nich.Ve~čtvrté kapitole je uveden návrh technického řešení pro~výpočet zpoždění.

V~projektové části, v~páté kapitole, je uveden postup prací připojení na~datová rozhraní.
Šestá kapitola obsahuje popis distribuce zpoždění.


% ============================================================================ %
\cast{Teoretická část}

\n{1}{Vývojové nástroje, protokoly a aplikace}
Pro~vývoj na~této diplomové práci bude použito mnoho různých framworků, komunikačních protokolů a~jejich znalost je klíčová k~úspěšnému dokončení projektu.  
\begin{itemize}
	\setlength{\parskip}{0pt}
	\setlength{\itemsep}{0pt}
	\setstretch{1.05}
	\item {Spring Framework}
	\item {Hibernate~framework}
	\item {Apache~Maven}
	\item {SOAP}
	\item {WSDL}
	\item {SoapUI}
	\item {REST}
	\item {JSON}
	\item {Postman}
	\item {JavaMail}
	\item {CSV}
	\item {PostGIS}
\end{itemize}
\n{2}{Spring}
Spring Framework~\cite{spring-framework} (dále jen spring) je populární open-source~\cite{open-source} pro~vývoj J2EE~\cite{j2ee} aplikací. Jádro springu~\cite{spring-in-action} je využívá návrhový vzor IoC (Inversion~of~Control~\cite{ioc}) a~je označován jako IoC~kontejner. Tento návrhový vzor funguje na~principu přesunutí zodpovědnosti za~vytvoření a~provázání objektů z~aplikace na~framework.

Objekty lze získat prostřednictvím Dependency~Injection~\cite{dependency-injection} neboli vsazování závislostí (Obr.~\ref{fig:dependency-injection}). Jedná se o~speciální případ IoC. Dependency Injection lze realizovat třemi způsoby
\begin{itemize}
	\setlength{\parskip}{0pt}
	\setlength{\itemsep}{0pt}
	\setstretch{1.05}
	\item {Setter Injection,}
	\item {Constructor Injection,}
	\item {Interface Injection.}
\end{itemize}
\obr{Náhled možností dependency injection}{fig:dependency-injection}{0.75}{graphics/dependency-injection.png}

Vytvářené objekty se nazývají JavaBeans~\cite{java-beans}. Ty jsou vytvářeny typicky na~základě načtení konfiguračního souboru ve~formátu XML~\cite{xml}, který obsahuje definice těchto Beans. Spring se nezabývá řešením již vyřešených problémů. Místo toho využívá prověřených a~dobře fungujících existujících open-source nástrojů, které v~sobě integruje. Tím se stává jejich použití často jednodušším. Spring je modulární framework. Umožňuje využít jen část, která se zrovna hodí k~řešení daného problému. Účelem springu je:
\begin{itemize}
	\setlength{\parskip}{0pt}
	\setlength{\itemsep}{0pt}
	\setstretch{1.05}
	\item {zjednodušení návrhu J2EE aplikací se zaměřením na~architekturu aplikace (místo na~technologie),}
	\item {jednoduchá testovatelnost,}
	\item {neinvazivní rozvoj a modularita.}
\end{itemize}

\n{2}{Hibernate}
Hibernate~framework~\cite{hibernate-framework} je napsaný v~jazyce Java~\cite{java}, který umožňuje tzv. objektově-relační mapování (ORM~\cite{orm}). Usnadňuje řešení otázky zachování dat objektů i~po~ukončení běhu aplikace.Hibernate poskytuje způsob, pomocí něhož je možné zachovat stav objektů mezi dvěma spuštěnými aplikacemi. Říkáme tedy, že udržuje data persistentní. Dosahuje toho pomocí ORM, což znamená, že mapuje Javovské objekty na~entity v~relační databázi. K~tomu používá tzv. mapovací soubory, ve~kterých je popsáno, jakým způsobem se mají data z~objektu transformovat do~databáze a~naopak, a~jakým způsobem se z~databázových tabulek mají vytvořit objekty (Obr.~\ref{fig:hibernate-diagram}).

\obr{Příklad integrace Hibernate do projektu~\cite{hibernate-tutorial}} {fig:hibernate-diagram}{0.6}{graphics/hibernate-diagram.jpg}

Druhý způsob jak mapovat objekty je použít anotace místo mapovacích souborů. V~Hibernate se tedy pracuje s~běžnými business objekty, přičemž mohou být sloupce tabulky spojeny přímo s~atributy objektu, nebo mohou být připojeny skrze metody get/set a~metody hashCode() a~equals(). Nutno podotknout, že nelze použít EJB~\cite{enterprise-javabeans} (viz JavaBean), ale pouze klasické objekty - tzv. POJO (Plain Old Java Object)~\cite{pojo}. Poté, co jsou objekty uložené v~databázi, se na~ně lze dotazovat jazykem HQL (Hibernate Query Language), který je odvozen z~SQL~\cite{sql} a~je mu tedy velice podobný.

\n{2}{Maven}
Apache~Maven~\cite{apache-maven} je nástroj pro~správu, řízení a~automatizaci sestavení aplikace. Ačkoliv je možné použít tento nástroj pro projekty psané v~různých programovacích jazycích, podporován je především jazyk Java. Základním principem fungování Mavenu je popsání projektu pomocí POM. 

\obr{Příklad integrace Maven repository~\cite{maven-repositories}}{fig:maven}{0.7}{graphics/maven-repositories.png}

Tento model popisuje softwarový projekt nejen z~pohledu jeho zdrojového kódu, ale včetně závislostí na~externích knihovnách, popisu procesu buildování a~různých funkcí s~tím spojených (jako je spouštění testů, sbírání informací o~zdrojových kódech a~podobně). Maven~\cite{maven-lit} sám je postaven na~modulární architektuře a~funguje na~principu volání jednotlivých pluginů. Maven pouze obstarává dodání a~spuštění nadefinovaných pluginů. Maven nemá žádné vlastní grafické uživatelské rozhraní a~běží pouze na~příkazové řádce a~pluginy tak mohou využívat všechny nástroje, které dokáží komunikovat pomocí standardních vstupů. 

POM soubor poskytuje všechny konfigurace pro~jeden projekt. Za~tímto účelem je definovaná jednoduchá XML struktura, která definuje jednotlivé části projektu a~jeho závislosti na~externích knihovnách a~nástrojích. Současně je možné definovat konstanty, které pak mohou využít jednotlivé pluginy. Tento XML dokument se nachází v~kořenovém adresáři projektu a~je pojmenován pom.xml. Pokud je projekt složen z~více dílčích projektů nebo modulů, každý z~nich má pak svůj vlastní pom.xml soubor, který dědí vlastnosti od~nadřazeného souboru a~může přidávat další položky. Díky této struktuře je pak možné sestavit celý projekt jediným příkazem. V~pom.xml je možné u~každého projektu nadefinovat jeho závislosti na~externích knihovnách.

Jednotlivé prvky Artifacts jsou jednoznačně definovány podle atributů viz. (Kod.~\ref{pomExample.xml}). Maven pak automaticky vyhledá a~nainstaluje potřebné knihovny. Samotné vyhledávání probíhá v~definovaných úložištích (repository). Kromě globální maven repository, která je veřejně přístupná, je možné založit i~další soukromá nebo firemní úložiště.

\script{pomExample.xml}{xml}

\n{2}{SOAP}
SOAP~\cite{soap} je protokolem pro~výměnu zpráv založených na~XML přes~síť, hlavně pomocí HTTP~\cite{http}. Formát SOAP tvoří základní vrstvu komunikace mezi webovými službami a~poskytuje prostředí pro~tvorbu složitější komunikace. Existuje několik různých druhů šablon pro komunikaci na~protokolu SOAP. Nejznámější z~nich je RPC šablona, kde jeden z~účastníků komunikace je klient a~na~druhé straně je server. Server ihned odpovídá na~požadavky klienta.

\n{3}{WSDL}
WSDL~\cite{wsdl} je jazykem pro~popis funkcí, jež nabízí tzv. webová služba, a~dále pro~popis vstupů~a výstupů těchto funkcí (jinými slovy, co~webová služba poskytuje a~jak si o~to říci). Jelikož webová služba v~principu komunikuje protokolem SOAP, WSDL zpravidla popisuje SOAP komunikaci. WSDL vychází z~formátu XML. Podporované operace a~zprávy jsou popsány abstraktně, a~potom se omezují na~konkrétní síťový protokol a~formát zprávy. Z~toho plyne, že WSDL popisuje veřejné rozhraní webové služby.

\n{3}{SoapUI}
SoapUI~\cite{soapui} je open source aplikace pro~testování webových služeb pro~architektury orientované na~služby (SOA~\cite{soa}) a~reprezentace státních přenosů (REST). Její funkce zahrnují kontrolu webových služeb od~provolání přes~testování funkčnosti až~po~testování výsledků dotazů na~shodu s~očekávaným výsledkem.

Komerční verze SoapUI~Pro, která se zaměřuje hlavně na~funkce určené ke~zvýšení produktivity, byla také vyvinuta softwarem Eviware. V~roce 2011 společnost SmartBear Software získala produkt Eviware. SoapUI byl zpočátku propuštěn do~společnosti SourceForge v~září 2005. Je to~svobodný software, licencovaný na~základě podmínek veřejné licence Evropské unie. 

Je postavena výhradně na~platformě Java a~používá rozhraní Swing pro~uživatelské rozhraní. To znamená, že~SoapUI je multiplatformní. Dnes SoapUI podporují všechny hlavní vývojová prostředí (Obr.~\ref{fig:soap-ui}) (IDE~\cite{ide}), jako je např. IDEA, Eclipse nebo NetBeans a~spousta dalších. SoapUI může testovat webové služby SOAP a~REST, JMS~\cite{jms}, AMF~\cite{amf}, stejně jako volání HTTP(S) a~JDBC~\cite{jdbc}.

\obr{Podporované protokoly SOAP-UI~\cite{soap-ui}}{fig:soap-ui}{0.7}{graphics/soap-ui.png}

\begin{itemize}
	\setlength{\parskip}{0pt}
	\setlength{\itemsep}{0pt}
	\setstretch{1.05}
	\item {Výhody}
	\begin{itemize}
		\setlength{\parskip}{0pt}
		\setlength{\itemsep}{0pt}
		\setstretch{1.05}
		\item {je jednoduše čitelnější pro~člověka.}
	\end{itemize}	
	
	\item {Nevýhody}
	\begin{itemize}
		\setlength{\parskip}{0pt}
		\setlength{\itemsep}{0pt}
		\setstretch{1.05}
		\item {velký zápis komunikace,}
		\item {složitost,}
		\item {pomalé zpracování jednotlivými systémy (složité na~parsovaní a~validaci).}
	\end{itemize}
\end{itemize}

\n{2}{REST}
REST~\cite{rest} je architektura rozhraní, navržená pro~distribuovaná prostředí. REST navrhl a~popsal v~roce 2000 Roy Fielding (jeden ze spoluautorů protokolu HTTP~\cite{http}) v~rámci disertační práce Architectural Styles and the Design of Network-based Software Architectures~\cite{rest-publication}. Rozhraní REST je použitelné pro~jednotný a~snadný přístup ke~zdrojům (resources). Zdrojem mohou být data, stejně jako stavy aplikace (pokud je lze popsat konkrétními daty). REST je tedy na~rozdíl od~známějších XML-RPC či SOAP, orientován datově, nikoli procedurálně. Všechny zdroje mají vlastní identifikátor URI a~REST definuje čtyři základní metody pro přístup k~nim, které jsou známé pod označením CRUD:
\begin{itemize}
	\setlength{\parskip}{0pt}
	\setlength{\itemsep}{0pt}
	\setstretch{1.05}
	\item {\textbf{C}reate - vytvoření dat,}
	\item {\textbf{R}etrieve - získání požadovaných dat,}
	\item {\textbf{U}pdate - změnu,}
	\item {\textbf{D}elete - smazání.}
\end{itemize} 
Tyto metody jsou implementovány pomocí odpovídajících metod HTTP protokolu.

\n{3}{JSON}
JSON~\cite{json} je způsob zápisu dat (datový formát) nezávislý na~počítačové platformě, určený pro~přenos dat, který může být organizován v~polích nebo agregován v~objektech. Vstupem je libovolná datová struktura (číslo, řetězec, boolean, objekt nebo z~nich složené pole), výstupem je vždy řetězec. Složitost hierarchie vstupní proměnné není teoreticky nijak omezena. 

Kolekce párů \uv{název/hodnota} bývá v~rozličných jazycích realizována jako objekt, záznam (record), struktura (struct), slovník (dictionary), hash tabulka, klíčový seznam (keyed list) nebo asociativní pole.

Seřazený seznam hodnot je ve~většině jazyků realizován jako pole, vektor, seznam (list) nebo posloupnost (sequence).

Jedná se o~univerzální datové struktury a~v~podstatě všechny moderní programovací jazyky je v~nějaké formě podporují. Je tedy logické, aby na~nich byl založen i~na~jazyce nezávislý výměnný formát.
\begin{itemize}
	\item Object (Objekt) je uvozen znakem \uv{\{} (levá složená závorka) a~zakončen znakem \uv{\}} (pravá složená závorka). Každý název je následován znakem \uv{:} (dvojtečka) a~páry \uv{název/hodnota} jsou pak odděleny znakem \uv{,} (čárka).
	\item Array (Pole) je seřazenou kolekcí hodnot. Začíná znakem \uv{[} (levá hranatá závorka) a~končí znakem \uv{]} (pravá hranatá závorka). Hodnoty jsou odděleny znakem \uv{,} (čárka).
	\item Value (Hodnotou) rozumíme řetězec uzavřený do~dvojitých uvozovek, číslo, true, false, null, objekt nebo pole.
	Tyto struktury mohou být vnořovány.
	\item String (Řetězcem) je nula nebo více znaků kódování Unicode, uzavřených do~dvojitých uvozovek a~využívající únikových sekvencí (escape sequence) s~použitím zpětného lomítka. Znak je reprezentován jako řetězec s~jediným znakem. Řetězec je velmi podobný řetězcům z~jazyků C nebo Java.
	\item Number (Číslo) je podobné číslům z~jazyků C a~Java. Jedinou výjimkou je, že není používán oktalový ani hexadecimální zápis.
\end{itemize}

\n{3}{Postman}
Postman~\cite{postman} (Obr.~\ref{fig:postman}) je open source aplikace pro~testování webových služeb pro~architektury orientované na~REST.

\obr{Schéma užití Postman~\cite{postman}}{fig:postman}{1.0}{graphics/postman.png}

\n{2}{JavaMail}
Knihovna pro~příjímání a~odesíláni mailu. Javamail~\cite{javamail} je rozhraní jazyka Java, které pracuje na~základě protokolů SMTP, POP3 a~IMAP. JavaMail je integrován do~platformy Java~EE, ale nabízí i~volitelný balíček pro~použití v~Java~SE.

\n{2}{CSV}
CSV~\cite{csv} je jednoduchy souborový formát pro~výměnu tabulkových dat.
Soubor ve~formátu CSV sestává z~řádků, ve~kterých jsou jednotlivé položky odděleny znakem čárka \uv{,}.Hodnoty položek mohou být uzavřeny do~uvozovek \uv{"}, což umožňuje, aby~text položky obsahoval čárku. Pokud text položky obsahuje uvozovky, jsou pak zdvojeny.

\n{2}{PostGIS}
PostGIS~\cite{postgis} je open source software. Jedná se~o~nadstavbu pro~objektově-relační databázový systém PostgreSQL, která přidává podporu pro~geografické objekty~\cite{arcgis} (tzv. geoprvky). PostGIS~\cite{postgis2} implementuje specifikaci \uv{Simple Features for SQL} konsorcia Open Geospatial Consortium.


% ============================================================================ %

% Pokud Vaše práce neobsahuje analytickou část, stačí odstranit či zakomentovat nasledujících pár rádků
\cast{Analytická část}

\n{1}{Analýza obecně}
Analýza je vědecká metoda založená na~dekompozici celku na~elementární části, je to metoda
zkoumání složitějších skutečností rozkladem (dissolution) na~jednodušší. Cílem analýzy je tedy identifikovat podstatné a~nutné vlastnosti elementárních částí celku, poznat jejich podstatu a~zákonitosti. Analýza je také způsob výkladu, odděluje jednotlivé jevy a~zkoumá je izolovaně. Opačný postup k~analýze se nazývá syntéza. Cílem analýzy~u informačních systémů není pouze zdokumentování stávajícího stavu, ale zejména pochopení logiky fungování systému a~tím~pádem získání výchozích předpokladů pro~možnou optimalizaci.
Cílem analýzy tedy je:
\begin{itemize}
	\setlength{\parskip}{0pt}
	\setlength{\itemsep}{0pt}
	\setstretch{1.05}
	\item Získat znalosti o~systému
	\item Zjistit nedostatky a~slabá místa
	\item Uvědomit si potřebné změny
\end{itemize}
Analýza by měla postupovat od~globálního pohledu k~potřebným detailům.

\n{1}{Diferenční analýza}

\n{2}{Popis stávajícího stavu}
Stávající systém je založený na~komunikaci pomocí zasílání SMS mezi obsluhou vozu (řidič nebo steward) a~dispečerem (systém). SMS mají pevnou základní strukturu a~do systému jsou napojeny pomocí SMS brány (Gateway). Dále je možné zpoždění přímo zadat do~systému (Obr.~\ref{fig:addDelay}).
\obr{Vkládání zpoždění v~administrátorském systému}{fig:addDelay}{0.7}{graphics/addDelay.png}

Všechny tyto možnosti ukládání dat o~zpoždění jsou auditovány a~je možné zobrazovat jejich historii (Obr.~\ref{fig:auditDelay}).

\obr{Audit zpoždění}{fig:auditDelay}{0.45}{graphics/auditDelay.png}

\n{3}{L1 pohled}
\obr{Pohled L1 na současnou HW architekturu}{fig:HWpohled}{1.0}{graphics/HWpohled.png}

\n{2}{Popis plánovaného stavu}
Autobusy a~vlaky (vagony) budou vybaveny zařízením na~monitoring polohy (GPS).
Souřadnice budou odeslány a~zpracovány na~sběrném serveru. Tyto pak~budou vystaveny třetím stranám.
Napojení nového zdroje informací o~zpoždění (SŽDC). Napojení nového zdroje informací o~zastávkách a~stanicích (SŽDC). Vytvořit webové rozhraní pro~distribuci informací o~zpoždění.

\obr{Funkční požadavky}{fig:funkcniPozadavky}{1.0}{graphics/funkcniPozadavky.png}
\obr{Nefunkční požadavky}{fig:nefunkcniPozadavky}{0.7}{graphics/nefunkcniPozadavky.png}
\obr{Případy užití}{fig:useCase}{1.0}{graphics/useCase.png}

\n{2}{Určení rozdílu (mezery) mezi stávajícím a cílovým stavem}
Chybí napojeni na rozhrani SŽDC (GRAPP a Informační Tabule). Napojení na server Polohavozu.cz k načítaní GPS souřadnic. Načítaní struktury vlakových spojení. Doplnit nové funkce pro distribuci zpoždění.

\n{2}{Návrh variant dosažení cílového stavu (alternativní strategie)}
Spoje je nutné osadit technikou pro možnost sběru informací o poloze. Spoje je dále nutné napojit na servery agregující data o poloze. Agregovaná data musí být dále vyhodnocena pro~potřeby výpočtu zpoždění.

\n{2}{Popis stavu po~změnách}
I~nadále je využíván původní systém založený~na odesílání SMS s~zadaným textem a~hodnotou zpoždění. Nově jsou pak přidány zdroje dat o~poloze a~zpoždění spojů.

\n{1}{Navrhněte technické řešení pro~výpočet zpoždění}

\n{2}{Trasa spoje}
Trasu spoje lze popsat pomocí definovaného datového formátu KML (Obr.~\ref{fig:kml}). Zjednodušeně jde~o řadu GPS souřadnic, která popisuje konkrétní trasu spoje veřejné dopravy. Takto uchovaná data o~trase spoje jsou výhodná jak pro budoucí distribuci mapovým systémům třetích stran, tak jsou plně dostačující pro interní potřeby správy a~vizualizace spojena na mapovém podkladu.

\obr{Náhled spoje trasovaného pomocí KML na mapy.cz}{fig:kml}{1.0}{graphics/kml.jpg}

\n{2}{Definovat (základní) trasu spoje pomocí GPS souřadnic.}
Spoj se může za~mimořádných okolností vychýlit z~plánované trasy. Pro tyto účely je nezbytné mít nadefinovanou plánovanou trasu, popsanou pomocí GPS souřadnic s~dostatečnou hustotou pro následné vyhodnocení odchylky (jak v~prostoru, tak v~čase).

Pro snazší správu a~údržbu (nejen tras) je více než vhodné trasu rozdělit na~dílčí celky. Nabízí se rozdělit ji na~úseky mezi zastávkami (aby nebyl příliš velký počet souřadnic).

Realizace trasování spojů proběhne pomocí serveru polohavozu.cz, který agreguje data o poloze spojů. Tyto agregovaná data vystavuje pomocí REST API s dostatečnou propustností na~dotaz o~poloze spoje jednou za~sekundu. Spolu s~GPS bude uložen i~čas pořízení GPS, který odpovídá referenčnímu času vůči uražené vzdálenosti na~trase.

\n{2}{Výpočet zpoždění}
Na~základě referenčních hodnot a~aktuálních dat o~spoji z~provozu na~právě sledované trase již lze snadno získat zpoždění daného spoje. Při realizaci je nutné řešit problematiku jednak prostorového uspořádání GIS obecně, ale také nepřesnost užitých zařízení. Naměřená data je nutné nejprve normalizovat a~až následně použít jako vstup do mapových funkcí pro výpočet vzdálenosti od referenčního bodu. Využití PostGIS k~vracení nejbližší souřadnice tento výpočet velmi usnadňuje. Zároveň také poslouží pro~samotné uložení do~DB.

% ============================================================================ %
\cast{Projektová část}

\n{1}{Připojení nových zdrojů dat}
Pro zlepšení přesnosti stávající situace informací o~zpoždění bylo rozhodnuto o~napojení se na~nové zdroje informací, které nám pomohou zpřesnit a~hlavně automatizovat načítání, zpracovaní a~zobrazení informací o~zpoždění spojů.

\n{2}{Připojení SŽDC API}
Prvním z~nich je datové rozhraní GRAPP~(Obr.~\ref{fig:grapp}), ze~kterého se~bude načítat zpoždění a~poslední potvrzené průjezdové místo vozu pomocí kontrolního bodu. Druhým je datové rozhraní Informační tabule. Zde se budou načítat aktuální názvy nástupišť v~dané zastávce pro~daný spoj. Tato datová rozhraní spravuje společnost OLTIS group a.s., která jej provozuje pro~společnost SŽDC.

\obr{GRAPP~\cite{grapp}}{fig:grapp}{1.0}{graphics/grapp.png}

\n{3}{Datové rozhraní GRAPP}

\n{4}{Informační systém GRAPP}
, který vlastní společnost SŽDC, státní organizace (dále jen SŽDC), poskytuje datové rozhraní formou webové služby pro další systémy (poskytování dat z~informačního systému GRAPP třetím stranám). Aplikace třetích stran se mohou dotazovat do~informačního systému GRAPP na~aktuální data týkající se polohy vlaků veřejné osobní dopravy, respektive jejich poslední potvrzené polohy, na~dopravní síti SŽDC včetně doplňujících údajů.

\n{4}{Úprava pom.xml}
je první v~pořadí (Kod.~\ref{uprava-pom-szdc-position.xml}). Pom.xml je soubor, do~kterého je třeba doplnit novou část kódu v~sekci \textit{plugin}. Pomocí tohoto přidaného kódu pak~při~opětovném sestavení projektu (s~použitím Mavenu) dojde k~autogenerování tříd datového rozhraní GRAPP, které pak~využijeme pro~načítání dat.

\script{uprava-pom-szdc-position.xml}{xml}

\n{4}{Interface providera} se tvoří jako další krok (Kod.~\ref{SzdcProvider.java}) a~dále komunikační přepravky (Obr.~\ref{fig:szdc-position-models}).
 	
\obr{Vizualizace WSDL pro SŽDC (getTrain position)}{fig:szdc-position-models}{0.9}{graphics/szdc-position-models.jpg}
\script{SzdcProvider.java}{Java}

\n{4}{Tvorba service} je dalším krokem (Kod.~\ref{SzdcPositionWSApiService.java}), přes~které bude dotazováno na~datové rozhraní GRAPP.

\script{SzdcPositionWSApiService.java}{Java}

\n{4}{Implementace providera} a~jeho metod (Kod.~\ref{SzdcProviderImpl.java}), které budou využity.

\script{SzdcProviderImpl.java}{Java}

\n{4}{Vytvoření rozhraní a~implementace service} lze nyní provést (Kod.~\ref{SzdcService.java}) (Kod.~\ref{SzdcServiceImpl.java}) pro~načítání a~aktualizaci nového zdroje dat zpoždění. Pro~identifikaci spojení, na~které je nutné se dotázat datového rozhraní a~doplnit do requestu, je nutné vytvořit databázový dotaz. Ten bude realizován pomocí Hibernate (Kod.~\ref{SzdcDao.java}).

\script{SzdcDao.java}{Java}
\script{SzdcService.java}{Java}
\script{SzdcServiceImpl.java}{Java}

\n {4}{Automatické spouštění} je posledním krokem k~dokončení operace. Pomocí CRONu vytvoříme periodické spuštění dotazu na~datové rozhraní a~následné uložení do~databáze.

\script{applicationContex-cron-position.xml}{xml}


\n{3}{Datové rozhraní Informační tabule}
Další možností načítaní zdroje dat zpoždění je datové rozhraní informační tabule. Tato část implementace nakonec nebyla realizována. Její přínos pro nový dispečerský systém byl nulový. Toto rozhraní je primárně  využito k~načítaní informaci o~zastávkách a~nástupištích kudy spoj projíždí.

\n{4}{Informační systém TABULE}
, který vlastní společnost SŽDC, poskytuje datové rozhraní formou webové služby pro další systémy. Cílem je poskytování dat~z informačního systému TABULE pro~třetí strany. Aplikace třetích stran se mohou dotazovat do~informačního systému TABULE na~aktuální data týkající se odjezdových a~příjezdových tabulí dopravních bodů na~dopravní síti SŽDC, které mohou tato data poskytovat.

\n{2}{Připojení Polohavozu.cz}
V~první fázi byla napojena datová rozhraní od~SŽDC, bohužel neobsahovala všechna potřebná data. Chybějícím prvkem byla poloha, souřadnice GPS, potřebná k~určení, kde se vůz nachází a~pomoci ní vypočítat zpoždění.

\n{4}{Tvorba providera} (inteface, implementace)
	
\script{VehiclePositionProvider.java}{Java}
\script{VehiclePositionProviderImpl.java}{Java}
\script{PositionDO.java}{Java}
\script{TrackedVehiclePathDO.java}{Java}
\script{VehicleInfoDO.java}{Java}

\n{2}{Načítaní struktury vlaku do~dispečerského systému}
Pro správné dotazování se na~polohu vozů, je třeba nejprve získat strukturu složení vlaku. Jedna se o~aktuální zapojení vagonů. Načítá se bude identifikátor vagonu (jedinečná stringová hodnota). Takto identifikované vozy pak~jsou přiřazeny jednotlivým spojům a~uloženy do~databáze. Podle nich lze přesně získat data o~poloze.

\n{4}{Tvorba Provideru (inteface, implementace)}
	
\script{TrafficControlProvider.java}{Java}
	
\n{4}{Parsování CSV souboru a~uložení dat (konverze dat) do datových přepravek}
	
\script{TrafficControlProviderImpl.java}{Java}
	
\n{4}{Vytvoření datových přepravek}
	
\script{TrainCompositionDO.java}{Java}
\script{WagonDO.java}{Java}
	
\n{4}{JavaMail knihovna} slouží pro~napojení na vytvořenou emailovou schránku pomocí IMAP protokolu, odebírání excelových souboru se~strukturou vlaku na~aktuální den.
	
\script{TrafficControlMailDownloader.java}{Java}
	
Další oblastní zájmu je načtení přijatých a~odfiltrování přebytečných mailů, které jsou buď staré nebo neobsahuji potřebnou přílohu. Dále aplikační logika nezbytná k otevření přílohy a~načtení excelového souboru do streamu pro následné zpracování. Poslední částí zpracování excelu je konverze pracovních dat (vybraných buněk) do~formátu CSV.
	
\script{ExcelToCsvConverter.java}{Java}
	
\n{4}{Tvorba service} již není tak obsahově svázaná. Striktní definici podlého jen interface. Jeho implementace je již plně modifikovatelná. Lze využít např. funkcionálního inteface~z Java8~\cite{java8}. Přístup k~DB je odstíněný pomocí DAO. Lze plně využít stávajících funkcí pro~správu dat. Veškeré DB operace jsou zprostředkovány pomocí Hibenate.
	
\script{TrafficControlService.java}{Java}
\script{TrafficControlServiceImpl.java}{Java}
	
\n{4}{Tvorba CRON jobu} pro pravidelnou aktualizaci
	
\script{applicationContext-cron-Struktura.xml}{xml}
	

\n{1}{Distribuce zpoždění}
Vypočítané zpoždění spojů je vystaveno pomocí webové služby (REST) a~je konzumováno cílovými platformami. Základní platforma jsou Plazmy (současné digitální informační tabule na provozních místech Student Agency). Dále je zpoždění konzumováno splečností Google v rámci služby Google Transit (pro realizaci je použito napojení na~GTFS-real time). Zákazníkům, kteří mají povolené v~osobním nastavení notifikace (SMS případně email) chodí průběžné informace o~aktuálních změnách zpoždění spojů, na~které mají zakoupený lístek.

\n{2}{Zobrazení zpoždění spoje}
\begin{itemize}
	\setlength{\parskip}{0pt}
	\setlength{\itemsep}{0pt}
	\setstretch{1.05}
	\item Využít stávající funkci pro~zobrazování na~Plazmy
	\item Zasílaní SMS a~mailu o~zpoždění pro~jednotlivé prodejce a~dispečery podle skupin
\end{itemize}

\n{2}{Zobrazení údajů o zpoždění}
\begin{itemize}
	\setlength{\parskip}{0pt}
	\setlength{\itemsep}{0pt}
	\setstretch{1.05}
	\item Tvorba nového endpontu pro real-time GTFS
	\item GTFS-RT musí obsahovat stejné identifikátory jako statické GTFS
	\item Google Protokolbuffer
	\item Připojení načítání zpoždění a poloha GPS
\end{itemize}

% ============================================================================ %
\nn{Závěr}
Cílem této diplomové práce bylo analyzovat stávající řešení dispečinkového systému. Navrhnout nové rozšíření variant již existujícího modulu zpoždění. Vybrat ta nejlepší řešení a implementovat je do systému. Dále pak navrhnout technické řešení pro výpočet zpoždění a vytvořit datové rozhraní pro přístup třetích stran k údajům o zpoždění.

Z časových důvodu a přehodnocení priority úkolů nebyla dokončena část diplomové práce realizující vizualizaci spoje na trase.

Výsledkem práce je tedy rozšíření modulu zpoždění o nové zdroje dat a zpřenění zpoždění. K dalším rozšířením patří načítaní struktury vlakové soupravy a určení polohy jednotlivých vozů pomocí GPS. Všechny nové data jsou pak zpracovány a dal ditribuovány.

Řešení popsané v této diplomové práci ještě není kompletní a nabízí spostu možností úprav a vylepšení. To nejdůležitější jsou ale praktické zkušenosti a prohloubení znalostí při realizaci této diplomové práce.


% ============================================================================ %
