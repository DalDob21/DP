% ============================================================================ %
% Encoding: UTF-8 (žluťoučký kůň úpěl ďábelšké ódy)
% ============================================================================ %

% ============================================================================ %
\nn{Úvod}
První odstavec pod nadpisem se neodsazuje, ostatní ano (pouze první řádek, odsazení vertikální mezy odstavci je typycké pro anglickou sazbu; czech babel toto respektuje, netřeba do textu přidávat jakékoliv explicitní formátování, viz ukázka sazby tohoto textu s následujícím odstavcem).

Formátování druhého odstavce. Text text text text text text text text text text text text.


% ============================================================================ %
\cast{Teoretická část}

\n{1}{Vývojové nástroje, protokoly a aplikace}
text

\n{2}{Spring}
Spring Framework~\cite{spring-framework} (dále jen spring) je populární open-source~\cite{open-source} pro vývoj J2EE~\cite{j2ee} aplikací. Jádro springu je využívá návrhový vzor IoC (Inversion~of~Control~\cite{ioc}) a~je označován jako IoC~kontejner. Tento návrhový vzor funguje na~principu přesunutí zodpovědnosti za~vytvoření a~provázání objektů z~aplikace na~framework.

Objekty lze získat prostřednictvím Dependency~Injection~\cite{dependency-injection} neboli vsazování závislostí (Obr.~\ref{fig:dependency-injection}). Jedná se o~speciální případ IoC. Dependency Injection lze realizovat třemi způsoby
\begin{itemize}
	\setlength{\parskip}{0pt}
	\setlength{\itemsep}{0pt}
	\setstretch{1.05}
	\item {Setter Injection,}
	\item {Constructor Injection,}
	\item {Interface Injection.}
\end{itemize}

\obr{Náhled možností dependency injection}{fig:dependency-injection}{1.0}{graphics/dependency-injection.png}

Vytvářené objekty se nazývají JavaBeans~\cite{java-beans}. Ty jsou vytvářeny typicky na~základě načtení konfiguračního souboru ve~formátu XML~\cite{xml}, který obsahuje definice těchto Beans. Spring se nezabývá řešením již vyřešených problémů. Místo toho využívá prověřených a~dobře fungujících existujících open-source nástrojů, které v~sobě integruje. Tím se stává jejich použití často jednodušším. Spring je modulární framework. Umožňuje využít jen část, která se zrovna hodí k~řešení daného problému. Účelem springu je:
\begin{itemize}
	\setlength{\parskip}{0pt}
	\setlength{\itemsep}{0pt}
	\setstretch{1.05}
	\item {zjednodušení návrhu J2EE aplikací se zaměřením na~architekturu aplikace (místo na~technologie),}
	\item {jednoduchá testovatelnost,}
	\item {neinvazivní rozvoj a modularita.}
\end{itemize}

\n{2}{Hibernate}
Hibernate~framework~\cite{hibernate-framework} je napsaný v~jazyce Java~\cite{java}, který umožňuje tzv. objektově-relační mapování (ORM~\cite{orm}). Usnadňuje řešení otázky zachování dat objektů i~po~ukončení běhu aplikace.Hibernate poskytuje způsob, pomocí něhož je možné zachovat stav objektů mezi dvěma spuštěnými aplikacemi. Říkáme tedy, že udržuje data persistentní. Dosahuje toho pomocí ORM, což znamená, že mapuje Javovské objekty na entity v~relační databázi. K~tomu používá tzv. mapovací soubory, ve~kterých je popsáno, jakým způsobem se mají data z~objektu transformovat do~databáze a~naopak, a~jakým způsobem se z~databázových tabulek mají vytvořit objekty (Obr.~\ref{fig:hibernate-diagram}).

\obr{Příklad integrace Hibernate do projektu~\cite{hibernate-tutorial}} {fig:hibernate-diagram}{0.6}{graphics/hibernate-diagram.jpg}

Druhý způsob jak mapovat objekty je použít anotace místo mapovacích souborů. V Hibernate se tedy pracuje s~běžnými business objekty, přičemž mohou být sloupce tabulky spojeny přímo s~atributy objektu, nebo mohou být připojeny skrze metody get/set a metody hashCode() a equals(). Nutno podotknout, že nelze použít EJB~\cite{enterprise-javabeans} (viz JavaBean), ale pouze klasické objekty - tzv. POJO (Plain Old Java Object)~\cite{pojo}. Poté, co jsou objekty uložené v databázi, se na ně lze dotazovat jazykem HQL (Hibernate Query Language), který je odvozen z SQL~\cite{sql} a je mu tedy velice podobný.

\n{2}{Maven}
Apache~Maven~\cite{apache-maven} je nástroj pro správu, řízení a automatizaci buildů aplikací. Ačkoliv je možné použít tento nástroj pro projekty psané v~různých programovacích jazycích, podporován je převážně jazyk Java. Základním principem fungování Mavenu je popsání projektu pomocí POM. 

Tento model popisuje softwarový projekt nejen z pohledu jeho zdrojového kódu, ale včetně závislostí na externích knihovnách, popisu procesu buildování a~různých funkcí s~tím spojených (jako je spouštění testů, sbírání informací o zdrojových kódech a podobně). Maven sám je postaven na~modulární architektuře a funguje na~principu volání jednotlivých pluginů. Maven sám pouze obstarává dodání a spuštění nadefinovaných pluginů. Maven nemá žádné vlastní grafické uživatelské rozhraní a běží pouze na příkazové řádce a pluginy tak mohou využívat všechny nástroje, které dokáží komunikovat pomocí standardních vstupů. 

\textbf{Project Object Model koncept popisu projektu jako objektu.} Za~tímto účelem je definovaná jednoduchá XML struktura, která definuje jednotlivé části projektu a jeho závislosti na~externích knihovnách a nástrojích. Současně je možné definovat konstanty, které pak mohou využít jednotlivé pluginy. Tento XML dokument se nachází v~kořenovém adresáři projektu a je pojmenován pom.xml. Pokud je projekt složen z více dílčích projektů nebo modulů, každý z~nich má pak svůj vlastní pom.xml soubor, který dědí vlastnosti od nadřazeného souboru a může přidávat další položky. Díky této struktuře je pak možné sestavit celý projekt jediným příkazem. V pom.xml je možné u~každého projektu nadefinovat jeho závislosti na~externích knihovnách. 

Jednotlivé prvky Artifacts jsou jednoznačně definovány podle atributů <groupId> a <artifactId>. Maven pak automaticky vyhledá a nainstaluje potřebné knihovny. Samotné vyhledávání probíhá v~definovaných úložištích (repository). Kromě globální maven repository, která je veřejně přístupná, je možné založit i~další soukromá nebo firemní úložiště.

\n{2}{SOAP}
SOAP~\cite{soap} je protokolem pro výměnu zpráv založených na~XML přes síť, hlavně pomocí HTTP~\cite{http}. Formát SOAP tvoří základní vrstvu komunikace mezi webovými službami a poskytuje prostředí pro~tvorbu složitější komunikace. Existuje několik různých druhů šablon pro komunikaci na protokolu SOAP. Nejznámější z nich je RPC šablona, kde jeden z~účastníků komunikace je klient a na~druhé straně je server. Server ihned odpovídá na požadavky klienta.

\n{3}{WSDL}
WSDL~\cite{wsdl} je jazykem pro~popis funkcí, jež nabízí tzv. webová služba, a dále pro~popis vstupů a výstupů těchto funkcí (jinými slovy, co webová služba poskytuje a jak si o~to říci). Jelikož webová služba v~principu komunikuje protokolem SOAP, WSDL zpravidla popisuje SOAP komunikaci. WSDL vychází z~formátu XML. Podporované operace a~zprávy jsou popsány abstraktně, a~potom se omezují na~konkrétní síťový protokol a formát zprávy. Z toho plyne, že WSDL popisuje veřejné rozhraní webové služby.

\n{3}{SoapUI}
SoapUI~\cite{soapui} je open source aplikace pro~testování webových služeb pro~architektury orientované na~služby (SOA~\cite{soa}) a~reprezentace státních přenosů (REST). Její funkce zahrnují kontrolu webových služeb od~provolání přes testování funkčnosti až~po~testování výsledků dotazů na~shodu s~očekávaným výsledkem.

Komerční verze SoapUI~Pro, která se zaměřuje hlavně na~funkce určené ke~zvýšení produktivity, byla také vyvinuta softwarem Eviware. V~roce 2011 společnost SmartBear Software získala produkt Eviware. SoapUI byl zpočátku propuštěn do~společnosti SourceForge v~září 2005. Je to svobodný software, licencovaný na~základě podmínek veřejné licence Evropské unie. 

Je postavena výhradně na~platformě Java a~používá rozhraní Swing pro~uživatelské rozhraní. To znamená, že SoapUI je multiplatformní. Dnes SoapUI podporují všechny hlavní vývojová prostředí (IDE~\cite{ide}), jako je např. IDEA, Eclipse nebo NetBeans a spousta dalších. SoapUI může testovat webové služby SOAP a REST, JMS~\cite{jms}, AMF~\cite{amf}, stejně jako volání HTTP(S) a~JDBC~\cite{jdbc}.

\begin{itemize}
	\setlength{\parskip}{0pt}
	\setlength{\itemsep}{0pt}
	\setstretch{1.05}
	\item {Výhody}
	\begin{itemize}
		\setlength{\parskip}{0pt}
		\setlength{\itemsep}{0pt}
		\setstretch{1.05}
		\item {je jednoduše čitelnější pro člověka.}
	\end{itemize}	
	
	\item {Nevýhody}
	\begin{itemize}
		\setlength{\parskip}{0pt}
		\setlength{\itemsep}{0pt}
		\setstretch{1.05}
		\item {velký zápis komunikace,}
		\item {složitost,}
		\item {pomalé zpracování jednotlivými systémy (složité na parsovaní a validaci).}
	\end{itemize}
\end{itemize}

\n{2}{REST}
REST~\cite{rest} je architektura rozhraní, navržená pro~distribuovaná prostředí. REST navrhl a~popsal v~roce 2000 Roy Fielding (jeden ze spoluautorů protokolu HTTP~\cite{http}) v~rámci disertační práce Architectural Styles and the Design of Network-based Software Architectures~\cite{rest-publication}. Rozhraní REST je použitelné pro jednotný a snadný přístup ke zdrojům (resources). Zdrojem mohou být data, stejně jako stavy aplikace (pokud je lze popsat konkrétními daty). REST je tedy na rozdíl od známějších XML-RPC či SOAP, orientován datově, nikoli procedurálně. Všechny zdroje mají vlastní identifikátor URI a REST definuje čtyři základní metody pro přístup k nim, které jsou známé pod označením CRUD, tedy vytvoření dat (Create), získání požadovaných dat (Retrieve), změnu (Update) a smazání (Delete). Tyto metody jsou implementovány pomocí
odpovídajících metod HTTP protokolu.

\n{3}{JSON}
JSON je způsob zápisu dat (datový formát) nezávislý na počítačové platformě, určený pro přenos dat, která mohou být organizována v polích nebo agregována v objektech. Vstupem je libovolná datová struktura (číslo, řetězec, boolean, objekt nebo z nich složené pole), výstupem je vždy řetězec. Složitost hierarchie vstupní proměnné není teoreticky nijak omezena. Kolekce párů název/hodnota. Ta bývá v rozličných jazycích realizována jako objekt, záznam (record), struktura (struct), slovník (dictionary), hash tabulka, klíčový seznam (keyed list) nebo asociativní pole. Seřazený seznam hodnot. Ten je ve většině jazyků realizován jako pole, vektor, seznam (list) nebo posloupnost (sequence).
Jedná se o univerzální datové struktury a v podstatě všechny moderní programovací jazyky je v nějaké formě podporují. Je tedy logické, aby na nich byl založen i na jazyce nezávislý výměnný formát.
\begin{itemize}
	\item Object (Objekt)je uvozen znakem { (levá složená závorka) a zakončen znakem } (pravá složená závorka). Každý název je následován znakem : (dvojtečka) a páry název/hodnota jsou pak odděleny znakem , (čárka).
	\item Array (Pole) je seřazenou kolekcí hodnot. Začíná znakem [ (levá hranatá závorka) and končí znakem ] (pravá hranatá závorka). Hodnoty jsou odděleny znakem , (čárka).
	\item Value (Hodnotou) rozumíme řetezec uzavřený do dvojitých uvozovek, číslo, true, false, null, objekt nebo pole.
	Tyto struktury mohou být vnořovány.
	\item String (Řetězcem) je nula nebo více znaků kódování Unicode, uzavřených do dvojitých uvozovek a využívající únikových sekvencí (escape sequence) s použitím zpětného lomítka. Znak je reprezentován jako řetězec s jediným znakem. Řetězec je velmi podobný řetězcům z jazyků C nebo Java.
	\item Number (Číslo) je podobné číslům z jazyků C a Java. Jedinou výjimkou je, že není používán oktalový ani hexadecimální zápis.
\end{itemize}

\n{3}{Postman}
Postman je open source aplikace pro testování webových služeb pro architektury orentované na REST.

\n{2}{JavaMail}
Knihovna pro prijimani a odesilani mailu. Javamail je rozhraní API jazyka Java, které slouží k odesílání a přijímání e-mailů prostřednictvím protokolů SMTP, POP3 a IMAP. JavaMail je integrován do platformy Java EE, ale nabízí i volitelný balíček pro použití v Java SE.

\n{2}{CSV}
CSV je jednoduchy souborový formát pro výměnu tabulkových dat.
Soubor ve formátu CSV sestává z řádků, ve kterých jsou jednotlivé položky odděleny znakem čárka (,).
Hodnoty položek mohou být uzavřeny do uvozovek ("), což umožňuje, aby text položky obsahoval čárku.
Pokud text položky obsahuje uvozovky, jsou tyto zdvojeny.

\n{2}{PostGIS}
PostGIS je open source software. Jedná se o nadstavbu pro objektově-relační databázový systém PostgreSQL, která přidává podporu pro geografické objekty (tzv. geoprvky). PostGIS implementuje specifikaci „Simple Features for SQL“ konsorcia Open Geospatial Consortium.

\n{1}{Další nadpis}
Tato sekce obsahuje ukázku vložení obrázku (Obr. \ref{fig:logo}).

% Obrázek lze vkládat pomocí následujícího zjednodušeného stylu, nebo klasickým LaTex způsobem
% Pozor! Obrázek nesmí obsahovat alfa kanál (průhlednost). Jde to proti standardu PDF/A.
\obr{Popisek obrázku}{fig:logo}{0.5}{graphics/logo/fai_logo_cz.png}


\n{2}{Podnadpis}
Tato sekce obsahuje ukázku vložení tabulky (Tab. \ref{tab:priklad}).

% Tabulku lze vkládat pomocí následujícího zjednodušeného stylu, nebo klasickým LaTex způsobem
\tab{Popisek tabulky}{tab:priklad}{0.65}{|l|c|c|c|c|c|r|}{
  \hline
   & 1 & 2 & 3 & 4 & 5 & Cena [Kč] \\ \hline
  \emph{F} & (jedna) & (dva) & (tři) & (čtyři) & (pět) & 300 \\ \hline
}

\n{3}{Podpodnadpis}

\n{3}{Podpodnadpis}
Citace knihy. \cite{chmel}


% ============================================================================ %

% Pokud Vaše práce neobsahuje analytickou část, stačí odstranit či zakomentovat nasledujících pár rádků
\cast{Analytická část}

\n{1}{Analýza stávajícího dispečinkového systému}
Stavajici system zalozeny na komunikaci pomoci SMS mezi dispecerem a obsluhou vozu (ridic , steward). SMS se parsuji a ukladaji do databaze.

\n{2}{Podnadpis}
text

\n{1}{Navrhněte technické řešení pro výpočet zpoždění}
\begin{itemize}
	\item Vytvorime si zakladni trasu spoje pomoci GPS souradnic
	\item Tuto trasu rozdelime na trasy mezi zastavkami (aby nebyl prilis velky pocet souradnic)
	\item Pomoci Trasovani spoje (napojeni na polohavozu.cz) ziskame souradnice GPS a cas kdy byli tyto data porizeny
	\item Vytvorime modelovou trasu spoje s casem
	\item Porovanáním modelove trasy spoje a aktualni pozice vozu, muzem vypocist zpozdeni
	\item Vyuzijem PostGIS k vraceni nejblizsi souradnice, u teto souradnice zname i cas prujezdu
	\item Zpozdeni ulozime do databaze
\end{itemize}

\n{2}{Podnadpis}
text

% ============================================================================ %
\cast{Projektová část}

\n{1}{Připojení nových zdrojů dat}
Pro zlepšení přesnosti stavající situace informací o spoždění bylo rozhodnuto o napojení se na nové zdroje informací, které nám pomohou zpřesnit a hlavně automatizovat načítání, zpracovaní a zobrazení inforamcí o spoždění spojů.

\n{2}{Připojení SŽDC api}
 Prvním z nich je datové rozhraní GRAPP, ze ktere se bude načítat zpoždění a poslední potvrzené průjezdové místo vozu kontrolním bodem. Druhým je pak datové rozhraní Informační tabule, zde se budou načítat aktuální názvy nástupišť v dané zastávce pro daný spoj. Tyto datová rozhraní spravuje společnost Oltis group a.s. , která jej provozuje pro společnost SŽDC.
 
 \n{3}{Datové rozhraní GRAPP}
 Jako první upravýme pom.xml soubor do kterého přidáme v sekci <plugin> nové údaje. Pomocí tohoto přidaného kódu pak při opětovném sestavení projektu pomocí mavenu dojde k autovygenerování komunikačních přepravek, které pak vzužijem pro načítání dat.
 
 \obr{Úprava Pom.xml}{fig:pomXMLPosition}{1.0}{graphics/pomXMLPosition.png}
 
\begin{itemize}
	\item Tvorba Provideru (inteface, implementace)
	\item Vytvoreni wsdl factory trid, abstraktnich trid
	\item Tvorba dao vrstvy pro nacitani zastvaek a spoju podle typu spojeni a doby odjezdu (hibernate)
	\item Vytvoreni DO trid, datova struktura odpovedi  
	\item Konvertor na tyto DO třídy
	\item Tvorba service (inteface, implementace)
	\item Nacitani zpozdeni jednotlivych spoju
	\item Nacitani nazvu kolej/nastupiste podle doby prijezdu do zastvaky
	\item Tvorba cron jobu pro tyto servicy
\end{itemize}

\n{3}{Datové rozhraní Informační tabule}
text

\n{2}{Připojení Polohavozu.cz}
\begin{itemize}
	\item Tvorba Provideru (inteface, implementace)
	\item Pomoci Postman ziskame JSON fily ze kterych vytvorime POJO objekty
	\item dva endpointy
	\begin{itemize}
		\item seznam vozu a jejich identifikace
		\item GPS souradnice vozu obsahujici identifikator
	\end{itemize}
	\item Konverze na DO objekty
	\item tvorba cron jobu pro pravidelý update
	\item Dao ukladani namere polohy do DB
\end{itemize}

\n{3}{Podnadpis}
text

\n{2}{Načítaní struktury vlaku do dispečinkového systému}
\begin{itemize}
	\item Tvorba Provideru (inteface, implementace)
	\item Vytvoreni mailu pro odebírání excel souboru se strukturou vlaku rozepsanou na den
	\item Pripojeni se na tuto mailovou adresu pomocí javamail knihovny
	\item Nacteni prijatých mailu a odfiltrovani prebytecnych mailu, ktere jsou bud stare nebo neobsahuji prilohu
	\item Otevreni prilohy a nacteni exceloveho souboru
	\item Otevreni pracovniho sesitu a nacteni dat z exceloých bunek do CSV souboru
	\item Parsovani CSV souboru a ulozeni dat (konverze dat) do datových prepravek
	\item Dao funkce (hibenate dotaz do DB)
	\item Tvorba service (inteface, implementace)
	\item Pomocí stavajicích funkci ulozíme data do DB
	\item Finalni uprava a kontrolni mechanismy na sparvnost cislovani vozu
	\item tvorba cron jobu pro pravidelý update
\end{itemize}

\n{3}{Podnandpis}
text

\n{1}{Distribuce zpoždění}
Plazmy, GTFS-real time, SMS, mail

\n{2}{Zobrazení na zpozdeni spoje}
\begin{itemize}
	\item Vyuzit stavajici funkci pro zobrazovani na Plazmach
	\item Zasilani SMS a mailu o zpozdeni pro jednotlive prodejce a dispecery podle skupin
\end{itemize}

\n{3}{Podnandpis}
text

\n{2}{Zobrazení údajů o zpoždění}
\begin{itemize}
	\item Tvorba noveho endpontu pro real-time GTFS
	\item GTFS-RT musi obsahovat stejne identifikatory jako staticke GTFS
	\item Google Protokolbuffer
	\item Pripojeni nacitani zpozdeni a poloha GPS
\end{itemize}

\n{3}{Podnandpis}
text
% ============================================================================ %
\nn{Závěr}
Text závěru


% ============================================================================ %
